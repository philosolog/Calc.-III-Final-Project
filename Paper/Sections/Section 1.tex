\lesson{1}{}{The Extreme Value Theorem in $\mathbb{R}^2$}
% *: 
\setcounter{chapter}{0}
\chapter{Hungry Joe}
Our story begins with a random guy named "Joseph-Louis."
Because his name is kinda long, we'll just refer to him as "Joe."
Joe is pretty good at math, but he isn't really that good at making dietary choices.
Despite this, Joe wants to optimize the satisfaction he gets from every meal he eats.
Joe usually prefers vegetables and light snacks over meats.

Today, Joe is at Carl's Parlor (run by the arbitrarily named "Carl-Friedrich") in search for the maximum satisfaction he can get from the sweetness of ice cream.
Joe won't be satisfied enough if he has too little or too much ice cream.
In other words, he desires for a "Goldilocks" amount of sweetness today.
If he's only considering sweetness $(s)$ as a factor of his satisfaction, then his satisfaction $S$ can be described as:\par
\LARGE
\begin{equation}
	S(s) = 8e^{-\frac{(s-4)^2}{64}}
\end{equation}
\normalsize
\begin{eg}
	If Joe wants at least $1$ unit of sweetness and at most $5$ units, what is the maximum satisfaction that Joe can attain?
\end{eg}
% *: 
\setcounter{chapter}{1}
\chapter{Utilmaxxing}
\begin{theorem}[The Extreme Value Theorem in $\mathbb{R}^2$ - Paul's Online Notes ${[1]}$]
	Suppose that \(f\left( x \right)\) is continuous on the interval \(\left[ {a,b} \right]\) then there are two numbers \(a \le c,d \le b\) so that \(f\left( c \right)\) is an absolute maximum for the function and \(f\left( d \right)\) is an absolute minimum for the function.
\end{theorem}

Restated, the Extreme Value Theorem (EVT) in $\mathbb{R}^2$ guarantees for any absolute (global) maximum and minimum value for any closed-continuous interval on a function.
In differential calculus, this would imply that we need to compare the endpoints of the closed interval with the local extrema of the function.

Critical numbers (values) can represent either local extrema or inflection points, but we will regardlessly test for absolute extrema because an inflection point will always exist as values between the extrema.
The critical values of the function denote where the function's derivative equals $0$.
Keep this single variable function concept in mind, as it is important for later.
Let's help Joe find his maximum util!

Because $e^x$ is continuous for all $x$, we can apply the EVT to $S$ for $s\in[1, 5]$.
\begin{align*}
	S'(s)=-\frac{1}{4}(s-4)e^{-\frac{(s-4)^2}{64}}=0\\
	\implies e^{-\frac{(s-4)^2}{64}}=0\text{; }s-4=0
\end{align*}
Since $e^x > 0$ for all $x$, we can omit the first equation, giving us:
\begin{align*}
	s = 4
\end{align*}
Let's find its corresponding $S$ value...
\begin{align*}
	S(4) = 8e^{-\frac{(4-4)^2}{64}} = 8
\end{align*}
Now, we must test the $S$ values for the end values of the interval:
\begin{align*}
	S(1) = 8e^{-\frac{(1-4)^2}{64}} \approx 6.95052\\
	S(5) = 8e^{-\frac{(5-4)^2}{64}} \approx 7.87597
\end{align*}

Because $\max(S(1), S(4), S(5)) = S(4) = 8$, Joe can utilize the sweetness of ice cream to attain a maximum satisfaction of $8$ utils.