\lesson{1}{}{The Extreme Value Theorem in $\mathbb{R}^2$}

% *: 
\setcounter{chapter}{0}
\chapter{Hungry Joe}
The protagonist of our cliche math story is named "Joseph-Louis Lagrange."
 Because he is not really important, we'll refer to him as "Joe."
 Right now, he is critically hungry and craving bird meat.. so he goes to the local Burger King.


% *: 
\setcounter{chapter}{1}
\chapter{Utilmaxxing}
% Denote that optimization is a concept where the rate equals 0- and state its importance for later.

\begin{theorem}[$\dagger$ The Extreme Value Theorem in $\mathbb{R}^2$]
	Suppose that \(f\left( x \right)\) is continuous on the interval \(\left[ {a,b} \right]\) then there are two numbers \(a \le c,d \le b\) so that \(f\left( c \right)\) is an absolute maximum for the function and \(f\left( d \right)\) is an absolute minimum for the function.
\end{theorem}