\lesson{1}{}{The Extreme Value Theorem in $\mathbb{R}^2$}

% *: 
\setcounter{chapter}{0}
\chapter{Hungry Joe}
Our story begins with a random guy named "Joseph-Louis."
Because his name is kinda long, we'll just refer to him as "Joe."
Joe is pretty good at math, but he isn't really that good at making dietary choices.
Despite this, Joe wants to optimize the satisfaction he gets from every meal he eats.
Joe usually prefers vegetables and light snacks over meats.

Today, Joe is at Carl's Parlor (run by the arbitrarily named "Carl-Friedrich") in search for the maximum satisfaction he can get from the sweetness of icecream.
Joe won't be satisfied enough if he has too little or too much icecream.
He desires for his "Goldilocks" amount of sweetness today.
If he's only considering sweetness $(s)$ as a factor of his satisfaction, then his satisfaction $S$ can be described as:\par
\Large
\begin{equation}
	S(s) = 8e^{-\frac{(s-4)^2}{64}}
\end{equation}
\normalsize
\begin{eg}
	If Joe wants at least $1$ unit of sweetness and at most $5$ units, what is the maximum satisfaction that Joe can attain?
\end{eg}
% *: 
\setcounter{chapter}{1}
\chapter{Utilmaxxing}
% Denote that optimization is a concept where the derivative equals 0- and state its importance for later.

\begin{theorem}[$\dagger$ The Extreme Value Theorem in $\mathbb{R}^2$]
	Suppose that \(f\left( x \right)\) is continuous on the interval \(\left[ {a,b} \right]\) then there are two numbers \(a \le c,d \le b\) so that \(f\left( c \right)\) is an absolute maximum for the function and \(f\left( d \right)\) is an absolute minimum for the function.
\end{theorem}