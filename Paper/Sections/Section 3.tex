\lesson{3}{}{The Method of Lagrange Multipliers}

\setcounter{chapter}{4}
\chapter{Another Order} % TODO: #
Delightful! Joe greatly enjoyed the addition of meat- the piquant umami was a new experience for his buds.
Despite already heightening his satisfaction twice, Joe was yet again deciding on another combination of a meat-topped icecream.
This time, his only constraint was that he wanted the umami flavor to be inversely proportional to half the sweetness he exhibits.
\begin{eg} % TODO: Linked ref.
	Given that his satisfaction can again be represented by eq. (3.1), Joe desires for a nonnegative amount of each flavor and wants to try a combination where the flavor of umami he attains is inversely proportional to half the sweetness.

	With what combination of $(u, s)$ can Joe attain his maximum satisfaction?
\end{eg}
\setcounter{chapter}{5}
\chapter{A Joe Analysis}
% Here, show that LM can be applied but EVT cannot- with a geometric proof instead of the disgusting algebra.

\setcounter{chapter}{6}
\chapter{Metonymization, Part 1} % Is there a better to word "an off-example"?
Now, for what this project was all about, let's apply Joe's strategy to a generalized example.

% TODO: List common, useful, and interesting applications.. including ones that are to be mentioned in the slides, since presentations will be after we turn our paper in.