\lesson{4}{}{The Cobb-Douglas Production Function}
\setcounter{chapter}{7}
\chapter{Carl's Parlor}
After Joe decided to buy everything from Carl's Parlor, Carl-Friedrich decided to not accept anymore customers until the icecream and chicken dispensers were refilled.
Also, Carl was the only worker at his icecream shop, so he is interested in hiring more workers.

To financially plan these business plans, Carl viewed a couple YouTube videos on the \textbf{Cobb-Douglas production function}.
\begin{remark}
	The Cobb-Douglas production function is an economics concept that relates a firm's production output $Y$ in terms of two variable inputs, labor $(L)$ and capital $(K)$.

	The production function can be written as:
	\[Y(L, K) = AL^{\alpha}K^{1 - \alpha}\]
	Where..
	\begin{itemize}
		\item A is a constant for \textit{$^\ddagger$total-factor productivity},
		\item and $\alpha$ is a constant for \textit{$^\ddagger$output elasticity of labor}
	\end{itemize}
	If there are any constraints on the variables $L$ and $K$, like a budget, then the method of Lagrange multipliers can be utilized to find the specific quantities of capital and labor that maximizes output $Y(L, K)$ for constraint functions.

	$^\ddagger$ For the sake of keeping this paper focused on constraining the Cobb-Douglas production function, we will not derive these economic concepts.
\end{remark}

Right now, Carl is eyeing multi-purpose dispenser machines that are worth around $\$1,500$ and 
\[\]
\setcounter{chapter}{8}
\chapter{Money-Mouth Face Emoji}

\setcounter{chapter}{9}
\chapter{Metonymization, Part 2}