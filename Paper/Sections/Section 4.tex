\lesson{4}{}{The Cobb-Douglas Production Function}
% *: 
\setcounter{chapter}{8}
\chapter{Carl's Parlor}
After Joe decided to buy everything from Carl's Parlor, Carl-Friedrich decided to not accept anymore customers until the ice cream and chicken dispensers were refilled.
Also, Carl was the only worker at his ice cream shop, so he is interested in hiring more workers.

To financially plan these business plans, Carl viewed a couple YouTube videos on the \textbf{Cobb-Douglas production function}.
\begin{remark}
	The Cobb-Douglas production function is an economics concept that relates a firm's production output $Y$ in terms of two variable inputs, labor $(L)$ and capital $(K)$.

	The production function can be written as:
	\begin{center}
		$Y(L, K) = AL^{\alpha}K^{\beta}$
	\end{center}
	Where..
	\begin{itemize}
		\item A is a constant for \textit{$^\ddagger$total-factor productivity},
		\item $\alpha$ is a constant for \textit{$^\ddagger$output elasticity of labor},
		\item and $\beta$ is a constant for \textit{$^\ddagger$output elasticity of capital}.
	\end{itemize}

	$^\ddagger$ For the sake of keeping this paper focused on constraining the Cobb-Douglas production function, we will not derive these economic concepts.
	For the following example problems, we will make the following simplification: \[\beta = 1 - \alpha\]
	This derives from the economic concept of $^1$elasticity that we will only briefly describe in the footnotes.% TODO: Fact-check me on the terms lol...
\end{remark}

Right now, Carl is eyeing some refurbished multi-purpose dispenser machines with individual maintenance costs at around $\$1,500$ per year.
He is also looking to hire workers that will have to be paid $\$30,000$ a year.

Carl has rich parents, but he still has a budget.
He wants to spend no more than $\$100,000$ a year on his ice cream parlor.
In Carl's world, ice cream parlors can be found to have a production function similar to:\par
\LARGE
\begin{equation}
	Y(L, K) = L^{0.35}K^{0.65}
\end{equation}
\normalsize
\begin{eg}
	With the production function given above, Carl wants to find an optimal combination of capital and labor that maximizes his parlor's output while also considering his budget.

	Can you help him find such combination?
\end{eg}
% *: 
\setcounter{chapter}{9}
\chapter{Money-Mouth Face Emoji}
If there are any constraints on the variables $L$ and $K$, like a budget, then the method of Lagrange multipliers can be utilized to find the specific quantities of capital and labor that maximizes output $Y(L, K)$ for constraint functions.
% *: 
\setcounter{chapter}{10}
\chapter{Metonymization, Part 3}
As shown in the above example, the method of Lagrange multipliers can be applied to the Cobb-Douglas production function to find the specific quantities of labor and capital that maximizes production output for given constraints on both labor and capital.

To relate this to a generalized case, here's another example!
\begin{eg}
	Suppose you are given a specific Cobb-Douglas function \[f(x, y) = 50x^{0.4}y^{0.6}\] where $x$ is the dollar amount spent on labor and $y$ the dollar amount spent on equipment.
	
	Use the method of Lagrange multipliers to determine how much should be spend on labor and how much on equipment to maximize productivity if we have a total of 1.5 million dollars to invest in labor and equipment.
	
	We believe there could be a slight logic error in the example (see Ch. 12, Concluding Remarks). But because we were given this example problem as a requirement for this paper, we kept it ad litteram.
\end{eg}