\lesson{4}{}{The Cobb-Douglas Production Function}
% *: 
\setcounter{chapter}{8}
\chapter{Carl's Parlor}
After Joe decided to buy everything from Carl's Parlor, Carl-Friedrich decided to not accept anymore customers until the ice cream and chicken dispensers were refilled.
Also, Carl was the only worker at his ice cream shop, so he is interested in hiring more workers.

To financially plan these business plans, Carl viewed a couple YouTube videos on the \textbf{Cobb-Douglas production function}.
\begin{remark}
	The Cobb-Douglas production function is an economics concept that relates a firm's production output $Y$ in terms of two variable inputs, labor $(L)$ and capital $(K)$.

	The production function can be written as:
	\begin{center}
		$Y(L, K) = AL^{\alpha}K^{\beta}$
	\end{center}
	Where..
	\begin{itemize}
		\item A is a constant for \textit{$^\dagger$total-factor productivity},
		\item $\alpha$ is a constant for \textit{$^\dagger$output elasticity of labor},
		\item and $\beta$ is a constant for \textit{$^\dagger$output elasticity of capital}.
	\end{itemize}

	$^\dagger$ For the sake of keeping this paper focused on constraining the Cobb-Douglas production function, we will not derive these economic concepts.
	For the following example problems, we will make the following simplification: \[\beta = 1 - \alpha\]
	This derives from the economic concept of elasticity and how $\alpha + \beta = 1$ in this context.
	Note that we will not further explain the derivation of this in our paper.
\end{remark}

Right now, Carl is eyeing some refurbished multi-purpose dispenser machines with individual maintenance costs at around $\$1,500$ per year.
He is also looking to hire workers that will have to be paid $\$30,000$ a year.

Carl has rich parents, but he still has a budget.
He wants to spend no more than $\$100,000$ a year on his ice cream parlor.
In Carl's world, ice cream parlors can be found to have a production function similar to:\par
\LARGE
\begin{equation}
	Y(L, K) = L^{0.35}K^{0.65}
\end{equation}
\normalsize
\begin{eg}
	With the production function given above, Carl wants to find an optimal combination of capital and labor that maximizes his parlor's output while also considering his budget.

	Can you help him find such combination?
\end{eg}
% *: 
\setcounter{chapter}{9}
\chapter{Money-Mouth Face Emoji}
If there are any constraints on the variables $L$ and $K$, like a budget, then the method of Lagrange multipliers can directly be utilized to find the specific quantities of capital and labor that maximizes output $Y(L, K)$ for constraint functions.

In this problem, the constraint function would be the budget.
The budget function would look something like this:
\begin{equation}
	30,000L + 1,500K = 100,000% ?: Proof as to why we don't need EVT/inequalities even with an inequality for maximization?
\end{equation}

If we treat L and K as nonnegative, we can see that these constraints form a closed and bounded region.
Since $Y$ is differentiable for all combinations of nonnegative inputs, we can apply EVT to show that there does exist an absolute extrema in the bounded region.

So, we can take a couple partial derivatives and set the equations up:
\begin{align*}
	0.35L^{-0.65}K^{0.65} = 30,000\lambda\\
	0.65L^{0.35}K^{-0.35} = 1,500\lambda\\
	30,000L + 1,500K = 100,000
\end{align*}
Solving the algebra with a calculator, we get an $(L, K)$ pair that maximizes $Y$ for the given budget:
\begin{align*}
	\left(1.16667, 43.33333\right)
\end{align*}
Because a unit of $L$ or $K$ in this example should be integers, we would round down to get the optimal combination for Carl.

In the end, Carl should $\boxed{\text{hire }1\text{ worker and buy }43\text{ dispensers}}$!% ?: Naive algo.?
% *: 
\setcounter{chapter}{10}
\chapter{Metonymization, Part 3}
As shown in the above example, the method of Lagrange multipliers can be applied to the Cobb-Douglas production function to find the specific quantities of labor and capital that maximizes production output for given constraints on both labor and capital.

To relate this to a more general case, here's another example!
\begin{eg}
	Suppose you are given a specific Cobb-Douglas function \[f(x, y) = 50x^{0.4}y^{0.6}\] where $x$ is the amount of labor units and $y$ is the amount of equipment units.
	You are also given that each unit of labor costs $1$ and each unit of equipment costs $1$.
	
	Use the method of Lagrange multipliers to determine how much should be spend on labor and how much on equipment to maximize productivity if we have a total of $1.5$ million dollars to invest in labor and equipment.
	
	We believe there could be a slight logic error in the Cobb-Douglas example that was stated as a requirement for our project (see Ch. 13).
\end{eg}

\pagebreak
First, we set the partial derivatives of the production function equal to their $\lambda \vec{\nabla}g(...)$ counterparts, where $g(...)$ represents our constraint/budget ($x + y = 1,500,000$):
\begin{align*}
	50(0.4)x^{-0.6}y^{0.6} = \lambda\\
	50(0.6)x^{0.4}y^{-0.4} = \lambda\\
	x + y = 1,500,000
\end{align*}
Without showing the algebra (to keep this paper from getting any longer), the algebra concludes to the following solution:
\begin{align*}
	x = 600,000\\
	y = 900,000
\end{align*}
Because a unit of labor or equipment costs $1$, we can maximize our Cobb-Douglas production output $f$ with $(x,y)=(600000, 900000)$.

In general, the Cobb-Douglas production function is used to determine the production output of a firm, given units of labor and capital.
This value is modified by the elasticities (efficacy of either labor or capital on the production output) of each labor and capital.

In optimizing the production function with given constraints, you can find a combination of labor and capital that suffices a constraint (like a budget) while maximizing the production output for such budget.
In conclusion, the mathematical method of Lagrange multipliers can help you perform constrained optimization on not only the Cobb-Douglas production function, but various other functions that have continuous partial derivatives.