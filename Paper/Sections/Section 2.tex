\lesson{2}{}{The Extreme Value Theorem in $\mathbb{R}^3$}

\setcounter{chapter}{2}
\chapter{Hungrier Joe}
Since Joe is a math wizard, he already mentally precomputed that he needed $4$ units of sweetness in order to achieve his maximum satisfaction of $8$ utils.
Because of this, Joe was more fixated on a far more troubling matter...

Like other icecream parlors, Carl's Parlor serves high-quality chicken strips as an icecream topping.
Unfortunately, that is the ONLY icecream topping at Carl's.

Joe ponders the most optimal combination of cotton candy icecream and chicken strips that will provide him with the maximum satisfaction.
Joe's satisfaction $S$ can now be represented in terms of sweetness $(s)$ and umami $(u)$ as:\par
\Large
\begin{equation}
	S(s, u) = 8e^{-\frac{(s-4)^2+(u-4)^2}{64}}
\end{equation}
\normalsize
\begin{eg}
	Joe desires for at least $0$ units of either taste and a total sum of tastes that does not exceed $16$ units. What is the maximum satisfaction that Joe can achieve?
\end{eg}

\setcounter{chapter}{3}
\chapter{Nerd Emoji} % From Mr. Barraza's class, lol... Also, add emojis here...
